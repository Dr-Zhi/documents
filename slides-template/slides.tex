\documentclass[CJK,notheorems,mathserif,table]{beamer}
\mode<presentation>
\useoutertheme[height=0.1\textwidth,width=0.15\textwidth,hideothersubsections]{sidebar}
\usecolortheme{whale}      % Outer color themes 
\usecolortheme{orchid}     % Inner color themes
\useinnertheme[shadow]{rounded} 
\setbeamercolor{sidebar}{bg=blue!50} 
%\setbeamercolor{background canvas}{bg=blue!9}
\setbeamertemplate{background canvas}[vertical shading][bottom=white,top=structure.fg!25] 
%\usefonttheme{serif}  
\setbeamertemplate{navigation symbols}{} 

\usepackage{CJK}
\usepackage{xcolor}
\usepackage{fontspec}
%\newfontfamily\zhfont[BoldFont=Microsoft YaHei]{Microsoft YaHei} %设置中文
%\newfontfamily\zhpunctfont{Microsoft YaHei} % 设置中文
%\setmainfont{Microsoft YaHei UI}           %这里设置英文衬线字体
\setmonofont{Menlo}                     % fixed width
%\setsansfont{Microsoft YaHei UI}               %英文无衬线字体
\usepackage{zhspacing}
%\zhspacing

% macros
\definecolor{darkgreen}{rgb}{0,0.5,0}
\definecolor{darkgrey}{rgb}{0.5, 0.5, 0.5}
\newcommand{\code}[1]{\textcolor{darkgreen}{\footnotesize #1}} %\texttt{\scriptsize #1}}
\newcommand{\blue}[1]{\textcolor{blue}{#1}}
\newcommand{\grey}[1]{\textcolor{darkgrey}{#1}}

\title[Navigation Tab Bar] % (optional, use only with long paper titles)
{Navigation Tab Bar} 
%\subtitle %{Include Only If Paper Has a Subtitle}

\author[Yanling Zhi]{Yanling Zhi}
% - Use the \inst{?} command only if the authors have different affiliation.
\institute[MicroStrategy Inc.] { % (optional, but mostly needed) 
  iOS Dev Team, MicroStrategy Inc.
}

\date[May 9, 2013] % (optional, should be abbreviation of conference name)
{May 9, 2013}

%\subject{Theoretical Computer Science} % This is only inserted into the PDF information catalog.

% If you have a file called "university-logo-filename.xxx", where xxx
% is a graphic format that can be processed by latex or pdflatex,
% resp., then you can add a logo as follows:
% \pgfdeclareimage[height=0.5cm]{university-logo}{university-logo-filename}
% \logo{\pgfuseimage{university-logo}}

% Delete this, if you do not want the table of contents to pop up at
% the beginning of each subsection:
%\AtBeginSection[] {
%  \begin{frame}<beamer>{Outline}
%    \tableofcontents[currentsection]
%  \end{frame}
%}

% If you wish to uncover everything in a step-wise fashion, uncomment
% the following command:
%\beamerdefaultoverlayspecification{<+->}

\begin{document}

\begin{frame}
  \titlepage
\end{frame}

\section{UI Implementation}

\phantomsection
\begin{frame}{Agenda}
\tableofcontents
\end{frame}

\begin{frame}{Basic Implementation}%{Subtitles are optional.}
  \begin{itemize}
  \item Basic Implementation: 
    \begin{itemize} 
    \item \code{UITabBar, UITableView} (system default)
    \item \code{NavigationTabBarController} 
    \item \code{NavigationListController} 
    \item $\cdots\cdots$
    \end{itemize}
  \item Why not use system default \code{UITabBarController}?
    \begin{itemize}
      %\item easy-to-use controller for tab bar, and add more button automatically, but
      \item Although easy-to-use, it lacks of customized functions for our app, e.g., 
      \begin{itemize}
        \item Customization of \code{UITabBar}
        \item The co-existence with other view controllers
      \end{itemize}
    \end{itemize}
  \end{itemize}
\end{frame}


\section{Acknowledge-ments}

\begin{frame}{Acknowledgements}
People who have also contributed a lot to this feature:
\begin{itemize}
  \item John Sheppard, Teyla Emmagan
  \item Rodney Mackay
  \item Elizabeth Weir
\end{itemize}
\end{frame}

\begin{frame}
\begin{center} 
  {\Huge Any Questions?}
\end{center}
\end{frame}

\end{document}
